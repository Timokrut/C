\documentclass[a4paper,14pt]{article}
\usepackage[utf8]{inputenc}
\usepackage[english,russian]{babel}
\usepackage{amsmath}
\usepackage{amssymb}
\usepackage{geometry}
\usepackage{hyperref}
\usepackage{times}
\geometry{top=2cm,bottom=2cm,left=3cm,right=1.5cm}

\begin{document}

% Титульный лист
\begin{titlepage}
\begin{center}
МИНИСТЕРСТВО НАУКИ И ВЫСШЕГО ОБРАЗОВАНИЯ РОССИЙСКОЙ ФЕДЕРАЦИИ\\
Федеральное государственное автономное образовательное учреждение высшего образования\\
«САНКТ-ПЕТЕРБУРГСКИЙ ГОСУДАРСТВЕННЫЙ УНИВЕРСИТЕТ АЭРОКОСМИЧЕСКОГО ПРИБОРОСТРОЕНИЯ»\\

\vspace{1cm}
КАФЕДРА № 25\\

\vspace{3cm}
\textbf{ПОЯСНИТЕЛЬНАЯ ЗАПИСКА
К КУРСОВОЙ РАБОТЕ}\\

\vspace{1cm}
\textbf{АЛГОРИТМ ФАНО}\\

\vspace{0.5cm}
ЗАЩИЩЕНА С ОЦЕНКОЙ\\

\vfill
РУКОВОДИТЕЛЬ\\
ст. препод, канд. техн. наук\\
\vspace{2cm}
\underline{подпись, дата} \hspace{5cm} \underline{инициалы, фамилия}\\

\vspace{3cm}
\textbf{ПОЯСНИТЕЛЬНАЯ ЗАПИСКА\\К КУРСОВОЙ РАБОТЕ}\\

\vspace{1cm}
АЛГОРИТМ ФАНО\\
\vspace{1cm}
по дисциплине: ОСНОВЫ ПРОГРАММИРОВАНИЯ\\

РАБОТУ ВЫПОЛНИЛ\\
СТУДЕНТ ГР. № 2352\\
\vspace{2cm}
\underline{Т.А. Потапов} \hspace{5cm} \underline{подпись, дата}\\

\vfill
Санкт-Петербург 2024
\end{center}
\end{titlepage}

% Оглавление
\tableofcontents
\newpage

% Постановка задачи
\section{Постановка задачи}
Задачей данной курсовой работы является разработка программы для кодирования и декодирования файлов с использованием алгоритма Шеннона-Фано. Программа должна предоставлять возможность архивировать одиночные текстовые файлы, сжимая их размер, а затем восстанавливать их до исходного состояния.

Программа реализует алгоритм Фано, который основывается на разбиении символов исходного текста на группы, исходя из их частот, и присвоении им уникальных префиксных кодов.

\subsection*{Свойства алгоритма Фано:}
\begin{itemize}
    \item Кодирование должно быть префиксным, чтобы обеспечить однозначность декодирования.
    \item Часто встречающиеся символы получают более короткие коды, реже встречающиеся — более длинные.
    \item Программа должна эффективно обрабатывать текст с произвольным количеством символов.
\end{itemize}

\textbf{Пример задачи:} Дан текстовый файл с содержимым \texttt{hello world}. Программа должна сжать его, создав закодированный файл и сопутствующий словарь кодов. После разархивации из сжатого файла должен быть восстановлен исходный текст.

\textbf{Литература:} М.Н. Аршинов, Л.Е. Садовский, Коды и математика, Издательство: «Наука», 1996.

% Описание алгоритма
\newpage
\section{Описание алгоритма}
\subsection{Основные идеи алгоритма}
Алгоритм Шеннона-Фано основан на следующем:
\begin{enumerate}
    \item Символы сортируются по убыванию частоты их появления.
    \item Множество символов разбивается на две группы с примерно равной суммарной частотой.
    \item К первой группе добавляется бит 0, ко второй — 1.
    \item Процесс повторяется рекурсивно для каждой группы, пока все символы не будут закодированы.
\end{enumerate}

\subsection{Подробное описание алгоритма}
\begin{enumerate}
    \item \textbf{Сбор статистики:} анализируется входной текст, и подсчитывается частота каждого символа.
    \item \textbf{Сортировка символов:} символы сортируются по частоте в порядке убывания.
    \item \textbf{Построение кодов:} применяется рекурсивный метод, описанный выше.
    \item \textbf{Создание словаря:} генерируется отображение символов в их коды.
    \item \textbf{Кодирование текста:} каждый символ текста заменяется соответствующим кодом.
    \item \textbf{Декодирование:} используя словарь, сжатый текст преобразуется обратно в исходный.
\end{enumerate}

\subsection{Пример выполнения алгоритма}
Шаги алгоритма для текста \texttt{abacabad}:
\begin{enumerate}
    \item Частоты символов: $a: 4, b: 2, c: 1, d: 1$.
    \item Сортировка: $a (4), b (2), c (1), d (1)$.
    \item Разбиение: \begin{itemize}
        \item Первая группа: $a (4) \to$ код 0.
        \item Вторая группа: $b (2), c (1), d (1) \to$ код 1.
    \end{itemize}
    \item Рекурсивное разбиение второй группы:
        \begin{itemize}
            \item $b (2) \to$ код 10.
            \item $c (1), d (1) \to$ код 11.
        \end{itemize}
    \item Итоговые коды: $a: 0, b: 10, c: 110, d: 111$.
    \item Кодирование текста: \texttt{abacabad} $\to 010011010111$.
\end{enumerate}

\newpage
% Инструкция пользователя
\section{Инструкция пользователя}
\textbf{Запуск программы:}
\begin{verbatim}
./ShanonFano <режим> <входной файл> <выходной файл>
\end{verbatim}
\begin{itemize}
    \item $<$режим$>$: encode для сжатия или decode для разархивации.
    \item $<$входной файл$>$: имя файла для обработки.
    \item $<$выходной файл$>$: имя файла для сохранения результата.
\end{itemize}

\textbf{Формат входного файла:}
\begin{itemize}
    \item \textbf{Режим encode:} Текстовый файл с любым содержимым. Поддерживаются пробелы и переносы строк.
    \item \textbf{Режим decode:} Бинарный файл с закодированными данными.
\end{itemize}

\textbf{Формат выходного файла:}
\begin{itemize}
    \item \textbf{Режим encode:} Бинарный файл с закодированными данными, текстовый файл со словарем символов.
    \item \textbf{Режим decode:} Текстовый файл с восстановленным содержимым.
\end{itemize}

\newpage
% Тестовые примеры
\section{Тестовые примеры}
\begin{enumerate}
    \item \textbf{Тест 1:}
    \begin{itemize}
        \item Ссылка на текст: \url{https://www.gutenberg.org/cache/epub/12299/pg12299.txt}
        \item Вес текста до архивации: 385 KB
        \item Вес текста после архивации: 221 KB
    \end{itemize}
    \item \textbf{Тест 2:}
    \begin{itemize}
        \item Ссылка на текст: \url{https://www.gutenberg.org/cache/epub/12299/pg12299.txt}
        \item Вес текста до архивации: 280 KB
        \item Вес текста после архивации: 169 KB
    \end{itemize}
    \item \textbf{Тест 3:}
    \begin{itemize}
        \item Ссылка на текст: \url{https://www.gutenberg.org/cache/epub/80/pg80.txt}
        \item Вес текста до архивации: 620 KB
        \item Вес текста после архивации: 371 KB
    \end{itemize}
\end{enumerate}

\newpage
% Список литературы
\section{Список литературы}
\begin{enumerate}
    \item М.Н. Аршинов, Л.Е. Садовский, Коды и математика, Издательство: «Наука», 1996.
    \item \textit{Сайт:} «Алгоритм Шеннона-Фано», 2012, \url{https://habr.com/ru/articles/137766/}
    \item \textit{Видео:} «Метод Шеннона-Фано», 2017, \url{https://www.youtube.com/watch?v=orbJosR-Cqk&ab_channel=RomanTsarev}
\end{enumerate}


\end{document}